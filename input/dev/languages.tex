\section[Introduction to Languages]{Introduction to Languages}
Front End languages are language that are used to give better user experince and user interface. These mainly include HTML, CSS, Javascript. Some Frameforks like Bootstrap are also used with these basic languages.
\subsection{HTML}
\begin{figure}[!ht]
\centering
\includegraphics[width=0.3\textwidth]{input/images/HTML.png}                   
\caption{HTML5 Logo}
\hspace{-1.5em}
\end{figure}
HyperText Markup Language, commonly referred to as HTML, is the standard markup language used to create web pages. Along with CSS, and JavaScript, HTML is a cornerstone technology, used by most websites to create visually engaging webpages, user interfaces for web applications, and user interfaces for many mobile applications. Web browsers can read HTML files and render them into visible or audible web pages. HTML describes the structure of a website semantically along with cues for presentation, making it a markup language, rather than a programming language.


HTML elements form the building blocks of all websites. HTML allows images and objects to be embedded and can be used to create interactive forms. It provides a means to create structured documents by denoting structural semantics for text such as headings, paragraphs, lists, links, quotes and other items.

\begin{verbatim}
<!DOCTYPE html>
<html>
  <head>
    <title>This is a title</title>
  </head>
  <body>
    <p>Hello world!</p>
  </body>
</html>

\end{verbatim}


\subsection{CSS}

\begin{figure}[!ht]
\centering
\includegraphics[width=0.3\textwidth]{input/images/CSS.jpg}                   
\caption{CSS logo}
\hspace{-1.5em}
\end{figure}
Cascading Style Sheets (CSS) is a style sheet language used for describing the presentation of a document written in a markup language.Although most often used to set the visual style of web pages and user interfaces written in HTML and XHTML, the language can be applied to any XML document, including plain XML, SVG and XUL, and is applicable to rendering in speech, or on other media. Along with HTML and JavaScript, CSS is a cornerstone technology used by most websites to create visually engaging webpages, user interfaces for web applications, and user interfaces for many mobile applications.


CSS is desgned primarily to enable the separation of document content from document presentation, including aspects such as the layout, colors, and fonts. This separation can improve content accessibility, provide more flexibility and control in the specification of presentation characteristics, enable multiple HTML pages to share formatting by specifying the relevant CSS in a separate .css file, and reduce complexity and repetition in the structural content, such as semantically insignificant tables that were widely used to format pages before consistent CSS rendering was available in all major browsers. CSS makes it possible to separate presentation instructions from the HTML content in a separate file or style section of the HTML file. For each matching HTML element, it provides a list of formatting instructions

\begin{verbatim}
p {
    color: red;
    text-align: center;
} 
\end{verbatim}
\begin{figure}[!ht]
\centering
\includegraphics[width=0.3\textwidth]{input/images/JS.png}
\caption{Javascript logo}
\hspace{-1.5em}
\end{figure}

JavaScript (/ˈdʒɑːvəˌskrɪpt/) is a high-level, dynamic, untyped, and interpreted programming language. It has been standardized in the ECMAScript language specification. Alongside HTML and CSS, it is one of the three essential technologies of World Wide Web content production; the majority of websites employ it and it is supported by all modern web browsers without plug-ins. JavaScript is prototype-based with first-class functions, making it a multi-paradigm language, supporting object-oriented, imperative, and functional programming styles. It has an API for working with text, arrays, dates and regular expressions, but does not include any I/O, such as networking, storage or graphics facilities, relying for these upon the host environment in which it is embedded.

\iffalse
\subsection{PHP}
\begin{figure}[!ht]
\centering
\includegraphics[width=0.3\textwidth]{input/images/php.png}
\caption{PHP logo}
\hspace{-1.5em}
\end{figure}
 {\bf What is PHP?}

    PHP is an acronym for "PHP: Hypertext Preprocessor"
    PHP is a widely-used, open source scripting language
    PHP scripts are executed on the server
    PHP is free to download and use

 {\bf What is a PHP File?}

    PHP files can contain text, HTML, CSS, JavaScript, and PHP code
    PHP code are executed on the server, and the result is returned to the browser as plain HTML
    PHP files have extension ".php"

 {\bf What Can PHP Do?}

    PHP can generate dynamic page content
    PHP can create, open, read, write, delete, and close files on the server
    PHP can collect form data
    PHP can send and receive cookies
    PHP can add, delete, modify data in your database
    PHP can be used to control user-access
    PHP can encrypt data

  {\bf Why PHP?}

    PHP runs on various platforms (Windows, Linux, Unix, Mac OS X, etc.)
    PHP is compatible with almost all servers used today (Apache, IIS, etc.)
    PHP supports a wide range of databases
    PHP is free. Download it from the official PHP resource: www.php.net
    PHP is easy to learn and runs efficiently on the server side
\fi
\subsection{CMake}

CMake is a language(for generator of build systems) with abstract build rules and gnu make is a dependency resolves that executes programs. Cmake takes information on how to build programs generates makefiles that build the program. 

Simple Program 

\begin{figure}[!ht]
\centering
\includegraphics[width=0.6\textwidth]{input/images/cgs/hello.png}                   
\caption{test.cpp file}
\hspace{-1.5em}
\end{figure}

\begin{figure}[!ht]
\centering
\includegraphics[width=0.6\textwidth]{input/images/cgs/cm.png}                   
\caption{cmake file}
\hspace{-1.5em}
\end{figure}

\subsection{Shell Scripting}
Normally shells are interactive. It means shell accept command from you (via keyy
board) and execute them. But if you use command one by one (sequence of 'n' numbb
er of commands) , the you can store this sequence of command to text file and tee
ll the shell to execute this text file instead of entering the commands. This iss
 know as shell script.
Shell script defined as series of command written in plain text file. Shell scrii
pt is just like batch file is MS-DOS but have more power than the MS-DOS batch ff
ile.
why to Write Shell Script ?
\begin{enumerate}
\item Shell script can take input from user, file and output them on screen.
\item Useful to create our own commands.
\item Save lots of time.
\item To automate some task of day today life.
\item System Administration part can be also automated.
\end{enumerate}
{ \bf Execute your script as syntax:}
\begin{verbatim}
chmod 755 your-script-name
sh your-script-name
./your-script-name
\end{verbatim}

